\documentclass[11pt]{article}

% Engine-specific settings
% pdftex:
\ifcsname pdfmatch\endcsname
    \usepackage[T1]{fontenc}
    \usepackage[utf8]{inputenc}
\fi
% xetex:
\ifcsname XeTeXinterchartoks\endcsname
    \usepackage{fontspec}
    \defaultfontfeatures{Ligatures=TeX}
\fi
% luatex:
\ifcsname directlua\endcsname
    \usepackage{fontspec}
\fi
% End engine-specific settings

\usepackage{lmodern}
\usepackage{amssymb,amsmath}
\usepackage{graphicx}
\usepackage{fullpage}
\usepackage[keeptemps=all, makestderr, usefamily={rust, rs}]{pythontex}
\fvset{breaklines}

\begin{document}


\section*{Rust (\texttt{rust})}

Inline:  \rust{format!("$3 + 5 = {}$", 3+5)}.  \rusts{$4 + 6 = !{4+6}$}.


\begin{rustcode}
println!("Hello from Rust!");
println!("Running command family ``{}'', session ``{}'', restart ``{}''.", rstex.family, rstex.session, rstex.restart);
\end{rustcode}



\begin{rustblock}
println!("{}\\endinput", 2.0_f64.powf(8.0_f64));
\end{rustblock}

Printed output:  \printpythontex.

\begin{rustsub}
\color{blue}
\begin{Verbatim}
2.0_f64.powf(8.0_f64) = !{2.0_f64.powf(8.0_f64)}
\end{Verbatim}
\end{rustsub}


\section*{Rust (\texttt{rs})}

Inline:  \rs{format!("$3 + 5 = {}$", 3+5)}.  \rss{$4 + 6 = !{4+6}$}.


\begin{rscode}
println!("Hello from Rust!");
println!("Running command family ``{}'', session ``{}'', restart ``{}''.", rstex.family, rstex.session, rstex.restart);
\end{rscode}



\begin{rsblock}
println!("{}\\endinput", 2.0_f64.powf(8.0_f64));
\end{rsblock}

Printed output:  \printpythontex.

\begin{rssub}
\color{blue}
\begin{Verbatim}
2.0_f64.powf(8.0_f64) = !{2.0_f64.powf(8.0_f64)}
\end{Verbatim}
\end{rssub}



\section*{Errors}

\begin{rustblock}[error]
println!("{}\\endinput", 2.0_f64.powf(8.0_f64);
\end{rustblock}

\stderrpythontex



\section*{File tracking}

\begin{rustblock}[file_io]
// Open and write to two files, to check created file tracking
let mut my_file1 = io::BufWriter::new(rstex.open("my_file1.txt", open_mode::NW)
                                           .unwrap());
writeln!(my_file1, "This is a text file.").unwrap();
writeln!(my_file1, "Some more text.").unwrap();
writeln!(my_file1, "3 + 2 = {}", 3 + 2).unwrap();
drop(my_file1);

let mut my_file2 = io::BufWriter::new(rstex.open("my_file2.txt", open_mode::NW)
                                           .unwrap());
writeln!(my_file2, "This is another text file.").unwrap();
writeln!(my_file2, "Some more text for another text file.").unwrap();
writeln!(my_file2, "5 + 8 = {}", 5 + 8).unwrap();
drop(my_file2);


// Open and read from LaTeX file, to check dependency tracking
let in_file = "rust_test.tex";
println!("Reading from file \"{}\"...", in_file);
for (n, line) in io::BufReader::new(rstex.open(in_file, open_mode::R)
                                         .unwrap())
                               .lines()
                               .map(Result::unwrap)
                               .enumerate()
{
    if n < 8 {
        println!("line {}: {}", n + 1, line);
    }
}
\end{rustblock}

Text written to stdout:

\printpythontex[verbatim]

\end{document}
